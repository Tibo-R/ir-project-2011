\documentclass[10pt, a4paper]{article}
\usepackage[scale=0.80]{geometry}

% Options possibles : 10pt, 11pt, 12pt (taille de la fonte)
%                     oneside, twoside (recto simple, recto-verso)
%                     draft, final (stade de développement)

\usepackage[utf8]{inputenc}   % LaTeX, comprends les accents !
\usepackage[T1]{fontenc}      % Police contenant les caractères français
\usepackage{geometry}         % Définir les marges
\usepackage[english]{babel}  % Placez ici une liste de langues, la
                              % dernière étant la langue principale
\usepackage{fancyhdr}
\usepackage{graphicx}
\usepackage{wrapfig}
\usepackage{caption}

\pagestyle{headings}        % Pour mettre des entêtes avec les titres
                              % des sections en haut de page
\pagestyle{fancy}
\fancyhf{}
%\renewcommand{\chaptermark}[1]{\markboth{#1}{}}
%\renewcommand{\sectionmark}[1]{\markright{#1}}
%\chead{Thibault Roucou - \textbf{Assignment 5} - Information Retrieval}
\fancyfoot[RE,RO]{\textbf{\thepage}}

\title{Recommendations in Social Networks based on a Multi Agent System}           % Les paramètres du titre : titre, auteur, date
\author{Rodrigo Doria Medina and Thibault Roucou \\ Master Human Media Interaction \\ University of Twente, Netherlands}
% \date{}                     % La date n'est pas requise (la date du
                              % jour de compilation est utilisée en son
			      % absence
			      
		 
\begin{document}


\maketitle                  % Faire un titre utilisant les données
                              % passées à \title, \author et \date
                              
\begin{abstract}

The exponential growth of the Internet implies that the amount of information people can consult is growing in the same way and the information spread across the network change quickly over time. The problem of information overload has been in the focus of recent research in computer science and a number of solutions have been suggested. A popular solution is the search engines, but so far, they lack of personalization since their give the same results for everyone and don't take into account the different interests or expectations of users. Another solution is recommendation systems. A lot of recommendation methods exists such as content-based, collaborative filtering, web mining-based and so on, but they are always lack of intelligence and are note able to manage the dinamicity of the network. This paper discusses different methods to create a more intelligent recommendation system using multi agents and able to provide users with up-to-date and suitable recommendations.

\end{abstract}	

%-------------------------------------------%
\section{Introduction}
%-------------------------------------------%

%-------------------------------------------%
\section{Methods}
%-------------------------------------------%

%-------------------------------------------%
\section{Results}
%-------------------------------------------%

%-------------------------------------------%
\section{Discussion}
%-------------------------------------------%



\end{document}

